% Format teze zasnovan je na paketu memoir
% http://tug.ctan.org/macros/latex/contrib/memoir/memman.pdf ili
% http://texdoc.net/texmf-dist/doc/latex/memoir/memman.pdf
% 
% Prilikom zadavanja klase memoir, navedenim opcijama se podešava 
% veličina slova (12pt) i jednostrano štampanje (oneside).
% Ove parametre možete menjati samo ako pravite nezvanične verzije
% mastera za privatnu upotrebu (na primer, u b5 varijanti ima smisla 
% smanjiti 
\documentclass[12pt,oneside]{memoir}

% Paket koji definiše sve specifičnosti mastera Matematičkog fakulteta
\usepackage[latinica]{matfmaster}
%
% Podrazumevano pismo je ćirilica.
%   Ako koristite pdflatex, a ne xetex, sav latinički tekst na srpskom jeziku
%   treba biti okružen sa \lat{...} ili \begin{latinica}...\end{latinica}.
%
% Opicija [latinica]:
%   ako želite da pišete latiniciom, dodajte opciju "latinica" tj.
%   prethodni paket uključite pomoću: \usepackage[latinica]{matfmaster}.
%   Ako koristite pdflatex, a ne xetex, sav ćirilički tekst treba biti
%   okružen sa \cir{...} ili \begin{cirilica}...\end{cirilica}.
%
% Opcija [biblatex]:
%   ako želite da koristite reference na više jezika i umesto paketa
%   bibtex da koristite BibLaTeX/Biber, dodajte opciju "biblatex" tj.
%   prethodni paket uključite pomoću: \usepackage[biblatex]{matfmaster}
%
% Opcija [b5paper]:
%   ako želite da napravite verziju teze u manjem (b5) formatu, navedite
%   opciju "b5paper", tj. prethodni paket uključite pomoću: 
%   \usepackage[b5paper]{matfmaster}. Tada ima smisla razmisliti o promeni
%   veličine slova (izmenom opcije 12pt na 11pt u \documentclass{memoir}).
%
% Naravno, opcije je moguće kombinovati.
% Npr. \usepackage[b5paper,biblatex]{matfmaster}

% Pomoćni paket koji generiše nasumičan tekst u kojem se javljaju sva slova
% azbuke (nema potrebe koristiti ovo u pravim disertacijama)
\usepackage{pangrami}

% Paket koji obezbeđuje ispravni prikaz ćiriličkih italik slova kada
% se koristi pdflatex. Zakomentarisati ako na sistemu koji koristite ovaj
% paket nije dostupan ili ako ne radi ispravno.
\usepackage{cmsrb}

% Ostali paketi koji se koriste u dokumentu
\usepackage{listings} % listing programskog koda

% Datoteka sa literaturom u BibTex tj. BibLaTeX/Biber formatu
\bib{matfmaster-primer}

% Ime kandidata na srpskom jeziku (u odabranom pismu)
\autor{Luka B. Đorović}
% Naslov teze na srpskom jeziku (u odabranom pismu)
\naslov{Analiza slučajeva upotrebe relacionih i kolonski orijentisanih nerelacionih baza podataka}
% Godina u kojoj je teza predana komisiji
\godina{2024}
% Ime i afilijacija mentora (u odabranom pismu)
\mentor{др Saša \textsc{Malkov}, vandredni profesor\\ Универзитет у Београду, Математички факултет}
% Ime i afilijacija prvog člana komisije (u odabranom pismu)
\komisijaA{др Ана \textsc{Анић}, ванредни професор\\ University of Disneyland, Недођија}
% Ime i afilijacija drugog člana komisije (u odabranom pismu)
\komisijaB{др Лаза \textsc{Лазић}, доцент\\ Универзитет у Београду, Математички факултет}
% Ime i afilijacija trećeg člana komisije (opciono)
% \komisijaC{}
% Ime i afilijacija četvrtog člana komisije (opciono)
% \komisijaD{}
% Datum odbrane (obrisati ili iskomentarisati narednu liniju ako datum odbrane nije poznat)
\datumodbrane{15. јануар 2016.}

% Apstrakt na srpskom jeziku (u odabranom pismu)
\apstr{}

% Ključne reči na srpskom jeziku (u odabranom pismu)
\kljucnereci{анализа, геометрија, алгебра, логика, рачунарство, астрономија}

\begin{document}
% ==============================================================================
% Uvodni deo teze
\frontmatter
% ==============================================================================
% Naslovna strana
\naslovna
% Strana sa podacima o mentoru i članovima komisije
\komisija
% Strana sa posvetom (u odabranom pismu)
\posveta{Ovaj rad posvećujem...}
% Strana sa podacima o disertaciji na srpskom jeziku
\apstrakt
% Sadržaj teze
\tableofcontents*

% ==============================================================================
% Glavni deo teze
\mainmatter
% ==============================================================================

% ------------------------------------------------------------------------------
\chapter{Uvod}
% ------------------------------------------------------------------------------

Podaci su najstabilniji deo svakog sistema. Oni su reprezentacija cinjenica, koncepata i instrukcija u formalizovanom stanju spremnom za dalju interakciju, interpretaciju ili obradu od strane korisnika ili mašine. Iako kroz svoju istoriju racunatstvo vazi za oblast koja uvodi nove tehnologije i alate neverovatnom brzinom to  nije slucaj za svaku njenu granu. Postoje koncepti koji se kroz istoriju nisu menjali, ili su se slabo menjali i prosirivali. Primera za to ima puno, a to su uglavnom neki univerzanli funkcionalni principi koji se prozimaju kroz racunarske masine, kompilatore, operativne sisteme, sisteme za upravljanje podacima itd. Kada je rec o istoriji sistema za upravljanje podacima, ona se moze podeliti na 3 faze. Na period pre 1970. i Codd-ovog clanka u kojem govori o konceptima Relacionog modelovanja podataka.  Zatim na period od 1970 do ranih 2000ih i neprikosnovene vladavine relacionih sistema. Treca faza bi bila period nakon ranih 2000ih kada je doslo do razvoja novih tehnologija pod grupnim imenom "NoSQL"  koje se fokusiraju na poznate probleme standardnih relacionih sistema. Kao sto se da zakljuciti, vrlo rano u istoriji racunarstva naislo se na potrebu za standardizacijom mehanizama za obradu podataka.

Pre 70ih rukovanje podacima svodilo se na cuvanje podataka na fajl sistemu operativnog sistema. Apolo sletanje na mesec desilo se u vreme kada nije postojao sistem koji rukuje vecom kolicinopm podataka, sto dodatno govori o velicini takovg poduhvata. Nakon 70ih na talasu Codd-ovog clanka kao i projekta "Sistem R" kao dokaz koncepta da je relacioni model o kojem je Codd pisao moguce implementirati, skoro svaki sistem sa trajnim cuvanjem podataka koristio je  relacioni model. Kao lideri komercijalnih proizvoda ovog tipa nametnuli su se IBM i Oracle sa svojim sistemima za upravljanje relacionih baza podataka.

Ipak XXI vek doveo je znacajne alternative u oblasti cuvanja podataka informacionih sistema. Digitalizacija, a samim tim i potreba za obradom vece kolicine podataka, podstakla je nastanak novih tehnologija koje su sluzile da u novonastalom okruzenju omoguce da informacioni sistemi mogu da odgovore na zahteve modernog doba. Problemi te prirode obicno se stavljaju pod grupno ime "problemi velikih podataka" (BigData problems).

Iako su pridosle tehnologije ucestvovale u resavanju tih problema, decenije vladavine  relacionih sistema za cuvanje podataka ostavile su dubok trag u praksama rada sa podacima,  i sa razlogom predstavljaju defakto standard i dan danas. 
Sistematizovanje ogromne kolicine fizickog prostora na kojem se podaci mogu cuvati i kasnije koristiti, kao i fleksibilnost strukture podataka sa kojima se radi jesu glavni problemi na koje su se fokusirale tehnologije nastale u NoSQL pokretu. To sa sobom nosi umanjenje stabilnosti i oslanjanja na razvijen ekosistem koju neke organizacije  usled striktne biznis logike ne mogu priustiti. 

Kolonski orijentisane baze podataka su jedna vazna grupa nerelacionih baza, nastale kao plod BigQuery clanka iz 2004. godine. Njihova glavna odlika je da se podaci organizuju tako da srodni podaci treba da budu blizu jedni drugih kako bi se nad njima mogli primeniti razni optimizacioni algoritmi koji dovode do efikasnijeg skladistenja podataka.

Iz navedenog se naslucuje da nijedan od navedenih koncepata ne prednjaci po defaultu, zato je bitno postojanje materijala koji se bave analizom slucajeva upotrebe navedenih tehnologija. Pored teorijske analize koja se moze pronaci u relevantnim javnim dokumentacijama korisno je imati i konkretne implementacije benchmark-a ciji se rezultati mogu koristiti da se na osnovu  njih povuku paralele sa potrebama konkretnih realnih sistema.

Cilj rada .... iz prijave teme

\cite{PostgresHistory}


% ------------------------------------------------------------------------------
\chapter{Modeli za upravljanje podacima}
\label{chp:razrada}
\section{Relacioni model}
\subsection{Opšte karakteristike}
Relacioni model je najpopularniji model za rad sa podacima. On podatke kao i veze izmedju njih predstavlja kroz skup relacija. Iako kao fundamentalna ideja iza relacionog modela stoji tabelarni prikaz podataka, sto uvecava njegovu intuitivnost, korisno je imati na umu formalnu terminologiju koja se koristi u ovakvim sistemima. Svaki red tabele se naziva n-torka. Svaka kolona tabele se zove atribut. Presek reda i kolone je vrednosna celija.

Cesto se javlja dilema oko razlike izmedju tabele i relacije. Tabela je siri pojam, a da bi jedna tabela ujedno bila relacija mora ispuniti sledece uslove: presek kolone i vrste mora predstavljati jedinstvenu vrednost (datum), Sve vrednosne celije jedne kolone pripadaju istom tipu, Svaka kolona ima jedinostveno ime, ne postoje dva identicna reda jedne tabele.

S obzirom da u okviru jedne tabele ne mogu postojati dva identicna reda, jasno je da je pogodno imati nametnutu proceduru koja ne dozvoljava takvu pojavu. U slucaju relacionih modela to predstavlja superkljuc tabele. Superkljuc tabele je kolona ili skup kolona za koje se garantuje da ne mogu uzimati identicne vrednosti za vise redova jedne tabele. Minimalni skup kolona koji predstavlja superkljuc naziva se kljuc kandidat. Svaka tabela ima barem jedan kljuc kandidat za koji nijedna vrednost ne moze biti nepostojeca i taj kljuc kandidat se naziva primarni kljuc. Strani kljuc je kolona ili skup kolona cije vrednosti predstavljaju referencu na odredjeni red neke druge tabele. On igra veliku ulogu u ocuvanju integriteta baze podataka o cemu ce biti reci u nastavku.
\subsection{Integritet relacionog modela}
Cuvanje integritea relacionog modela predstavlja cuvanje preciznosti i tacnosti podataka koji se cuvaju u bazi. Ono nudi mehanizme ocuvanja konzistentnosti podataka prilikom invazivnih operacija kao sto su dodavanje reda, izmena reda ili brisanje reda u tabeli. Postoji vise vrsta integriteta u relacionom modelu: integritet entiteta, integritet domena, integritet neposojece vrednosti i referencijalni integritet.
Integritet entiteta nalaze da se u tabelu ne moze uneti red koji kao primarni kljuc ima nepostojecu vrednost. Integritet domena namece shemu po kojoj svaka kolona moze uzimati vrednost iz unapred dodeljenih skupova vrednosti. Integritet nepostojece vrednosti se govori o eventulnim kolonama cije vretnosti ne mogu kao vrednost imatu nepostojecu vrednost kako se ne bi narusila uspostavljena biznis logika. S bozirom da su asocijacije izmedju relacija determinisane postojanjem ranije pomenutih strani kljuceva u okviru tabele, oni igraju bitnu ulogu u ocuvanju referencijalnog integriteta modela. Referencijalni integritet nalaze da se svaki strani kljuc jedne tabele mora poklapati sa nekim od primarnih kljuceva uparene relacije ili u nekim slucajevima kao vrednost ima nepostojecu vrednost.
\subsection{Normalizacija}
Normalizacja predstavlja jasno definisan proces odlucivanja o tome koji atribut iu relaciji treba da budu grupisani kako bi se izbegla pojava redundantnih podataka. Redundanti podaci zauzimaju prostor na disku i otezavaju odrzavanje sistema. Normalizacija je unapredjivanje  logickog dizajna sistema tako da umanjuje dupliranje podataka kao i postizanje inkonzistentosti kroz invazivne operacije nad podacima, ali ne po cenu ocuvanja integriteta baze. Teorija o normalizaciji se zasniva na konceptima normalnih formi. Odredjenoj relaciji se dodeljuje odredjena normalna forma ukoliko zadovoljava pravila vezana za tu normalnu formu. Trenutno postoji 5 definisanih normalnih formi.
\subsection{ACID}

ACID (Atomicity, Consistency, Isolation, Durability) svojstva služe kao garancija tačnosti i konzistetnosti podataka prilikom konkurentnom pristupu. 

Atomicity se može objasniti pravilom: Jedna transakcija se izvršava u celini ili se ne izvršava nijedan njen deo. U prevodu, dejstvo transakcije je nedeljivo. 

Durability garantuje da će kompletirana transakcija u slučaju prekida rada sistema pre nego što su izmene reflektovane na disk, biti upamćena i izvršena nakon restarta sistema. Svaka izmena se upisjue u log fajl pre nego što je reflektovana na disk, kako bi se operacije mogle poništiti u slučaju poništavanja transakcije.

Consistency se čuva od strane korisnika. Bitno je da korisnik koji pokreće transakciju vodi računa o tome da stanje podataka ostane u konzistentom stanju.

Isolation svojstvo nalaže da se transakcije međusobno izolovane tako da izvršavanje jedne transakcije ne može uticati na izvršavanje druge. Ovo se obezbeđuje pomoću scheduler-a od strane samog sistema za upravljanje bazom podataka.

\subsection{PostgreSQL}

PostgreSQL je trenutno jedan od nanaprednijih baza podataka otvorenog koda.  Nastao iz POSTGRES projekta, vođenog od strane
Profesora Majkla Stonebraker-a. Prolazio je kroz dosta faza, od POSTGRES-a preko Postgres95 da bi 1996. godine dobio ime koje je i danas aktuelno a to je PostgreSQL kako bi naglasio vezu koju je uspostavio sa najnovijim mogućnostima SQL-a. \cite{PostgresHistory}
\section{Kolonski-orijentisani model}
% ------------------------------------------------------------------------------
\subsection{Opšte karakteristike}
\cite{ColumnarOriented}
Susret sa Big Data problemima dovelo je do potreba za tabelama koje imaju ogroman broj kolona, i ogroman broj redova u okviru tih tabela. Jasno je da nam je za potrebe različitih analitika potreban različit skup kolona. Navedena problematika predstavlja jedan od uočenih problema relacionih modela, gde je samo izvršavanje upita podrazumevalo dohvatanje svih kolona jednog reda, gde bi se filtriranje nepotrebnih kolona izvršavalo nakon što su se sve kolone učitale u memoriju. Kolonski orijentisan model dizajniran je tako da ovakav problem izbegne i uz to donese i druga poboljšanja o kojima će biti reči u nastavku.

Kolonski orijentisan model podatke tabele ne skladišti po redovima, već po kolonama. U prevodu, sve vrednosti kolone svih redova skladište se jedna do druge, a na konkretnu vrednosnu ćeliju referiše se pomoću ključa konkretnog reda kao i kolone čiju vrednost želimo da pročitamo. Ovakav dizajn doveo je do toga da za dohvatanje određenog skupa kolona nema potrebe da čitamo sve vrednosti tog sloga, već je dovoljno da znamo konkretan ključ tog reda kao i imena kolona čije vrednosti želimo da pročitamo.

S obzriom na bliskost   podataka na disku, nameće se mogućnost kompresije podataka, a s obzirom da su ti slični podaci lokalizovani na disku nema potrebe za velikom količinom meta informacija o kompresiji, što ovaj model čini posebno pogodnim za njihovu primenu.

Kolonski orijentisan model kao i većina ostalih nerelacionih modela, nudi fleksibilnost sheme. To kao posledicu to da eventualna promena strukture podataka nece bitno uticati na unapred definisanu shemu, kao ni iziskivati migraciju podataka, kao sto bi to bio slucaj kod relacionog modela. Osim toga fleksibilnost sheme se ogleda i u tome sto je broj kolona jednog vektora neogranicen, sto daje dosta prostora za eksperimentisanje sa dizajnom baze. Primer toga kako ovakvo svojstvo modela može doprineti dostizanju prednosti pri analitičkom sistemu možete videti na slici SLIKA 1. 

\subsection{Popularni primenjivi algoritmi kompresije}
\cite{ColumnarOptimizations}
\subsubsection{Enkodiranje zasnovano na recniku}

Enkodiranje zasnovano na rečniku (Dictionary based encoding) jeste tehnika kompresije podataka koja se može primeniti na vrednosti jedne kolone. Najefikasnija je nad kolonama koje imaju mali skup mogućih vrednosti. Rade tako što se u memoriji sačuvaju sve moguće vrednosti te kolone i svakoj od njih se dodeli ključ. Veličina ključa je direktno zavisna od kardinalnosti skupa vrednosti te kolone. Svaki unos ili izemna vrednosti kolone u konsultaciji sa postojećim rečnikom radi enkodiranje pristigle vrednosti, a svako dohvatanje vrednosti radi dekodiranje sačuvane vrednosti. Ovim se izbegava smanjuje ponavljanje velikih podataka tako smanjujući potrebnu memoriju na disku. Uglavnom su pogodne primene nad kolonama sa statičkim i opisnim podacima koji se ponavljaju.

\subsubsection{Enkodiranje po broju ponavljanja}

Enkodiranje po broju ponavljanja (Run Length Encoding) je jednostavan mehanizam za kompresiju podataka pogodan za kolonski orijentisane baze podataka. Funkcioniše tako što kada se naiđe na vrednost koja se ponavlja, ne skladišti duplikate već sačuva tu vrednost jednom a dodatno kao meta informaciju prosledi koliko se ta vrednost ponavlja. Takav vid optimizacije najkorisnije je očigledno kada su vrednosti sortirane, a s obzirom da su vrednosti kolona u kolonski orijentisanim bazama jedna do druge to otvara prostor za ovaj vid kompresije podataka kako bi se umanjilo zauzeće prostora.

\subsubsection{Delta enkoding}
Delta enkoding je mehanizam za optimizaciju prostora baze podataka koji se zasniva na čuvanju razlike između objekata a ne celih vrednosti. Primera za upotrebu ima dosta a jedan od najčešćih je slučaj datumskih kolona, gde će nam referentna vrednost biti neki konkretan datum, a vrednosti ostalih kolona će biti čuvane kao razlika u odnosu na njega, te je očigledna velika količina prostora koja se čuva u takvom slučaju korišćenjem ovog vida optimizacije.
\subsection{BASE svojstva}
\subsection{HBase}


% ------------------------------------------------------------------------------
\chapter{Slučajevi upotrebe}
% ------------------------------------------------------------------------------
\section{Opis i sadržaj eksperimenta}
Analiza i uporedjivanje slucajeva upotrebe bice realizovani na osnovu teorijskih i prakticnih  izvora i istrazivanja. Svaki primer ce biti pracen eksperimentom koji ce se sastojati od izvrsavanja razlicitih vrsta postupaka. Kao platforma za realizaciju eksperimenata koriscen je host sa docker engine-om. Specifikacije Host-a data je na slici. Pokretanje svakog od eksperimenata je identicno. U okviru repozitorijuma nalazi se sav shell i java kod kao i uputstvo za pokretanje svakog od eksperimenta, zajedno sa deplojment dijagramom.

Svakom eksperimentu dodeljen je precizno definisan kontekst radi uspostavljanja potpune kontrole okruzenja u kojem se eksperiment realizuje. U svrhu definisanja konteksta eksperimenata delom su iskoriscene poznate specifikacije za benchmark bazi podataka. Konkretno za slucaj OLTP okruzenja konsultovana je TPC-C specifikacija, za slucaj OLAP okruzenja konsultovana je TPC-H specifikacija. Distribuirao okruzenje je izuzeto.

Analiza rezultata eksperimenta sprovodi se kroz vise faza. Prva faza je uporedjivanje slozenosti , sto arhitekturalne, sto shematske, realizacije konkretnog slucaja upotrebe kao i da li je konkretan slucaj upotrebe moguce realizovati sa postojecom tehnologijom. Druga faza je uporedjivanje efikasnosti, koja podrazumeva uporedjivanje vremena izvrsavanja programa.  Svaka od faza ce ukljucivati tekstualnu diskusiju, slike kao i druge graficke prikaze ukoliko su pogodni.

Kako bi se postigao dovoljan dokaz koncepta (eng. proof of concept), ali i doslednost modernom
vremenu, kategorije slucajeva upotrebe koji ce biti obuhvaceni su:
\begin{enumerate}
\item Onlajn transakciono procesiranje (OLTP)
\item Onlajn analiticko procesiranje (OLAP)
\item Primena u distribuiranom okruzenju
\end{enumerate}

Kako su za predstavnike izabrani PostgresSQL i HBase,  za rezultate merenja iz GLAVE 3 treba uzeti u obzir da implementacija navedenih koncepata nije opsta za sve Relacione sisteme kao ni za sve kolonski orijentisane baze podataka.
\section{Primena u online transakcionom procesiranju (OLTP)}

Online transakciono procesiranje obuhvata kratke, jednostavne, uchestale promene  na relativno malom skupu podataka. Primer koji cemo koristiti jeste uopstena transakcija korisnika gde sa jednog racuna treba da se prebaci novac na drugi racun.

Specifikacija PostgresSQL modela:

Specifikacija HBASE modela:

Rezultati:

\section{Primena u online analitičkom procesiranju (OLAP)}

OLAP procesiranje sacinjeno je od skoro iskljucivo citanja podataka. Upiti koji se koriste obicno imaju parametre, imaju visok nivo kompleksnosti i visok procenat podataka kojima pristupa.
Primer koji cemo koristiti jeste uopsten primer odrzavanja trgovinskog lanca koji ima skup musterija, proizvoda, dobavljaca,  narudzbina. 
Nas OLAP eksperiment ce se sastojati iz dohvatanja izvestaja o ukupnom kvanitetu, ceni nakon odbijanja poreza, prosecnom popustu za dati status stavke narudzbine.


Specifikacija Postgres modela:

Specifikacija HBASE modela:

Rezultati:

\section{Primena u distribuiranom okruženju}
\subsection{CAP teorema}

\chapter{Zaključak}

% ------------------------------------------------------------------------------
% Literatura
% ------------------------------------------------------------------------------
\literatura

% ==============================================================================
% Završni deo teze i prilozi
\backmatter
% ==============================================================================

% ------------------------------------------------------------------------------
% Biografija kandidata
\begin{biografija}
\textbf{Вук Стефановић Караџић} (\emph{Тршић, 26. октобар/6. новембар
  1787. — Беч, 7. фебруар 1864.}) био је српски филолог, реформатор
српског језика, сакупљач народних умотворина и писац првог речника
српског језика.  Вук је најзначајнија личност српске књижевности прве
половине XIX века. Стекао је и неколико почасних доктората.
Учествовао је у Првом српском устанку као писар и чиновник у
Неготинској крајини, а након слома устанка преселио се у Беч,
1813. године. Ту је упознао Јернеја Копитара, цензора словенских
књига, на чији је подстицај кренуо у прикупљање српских народних
песама, реформу ћирилице и борбу за увођење народног језика у српску
књижевност. Вуковим реформама у српски језик је уведен фонетски
правопис, а српски језик је потиснуо славеносрпски језик који је у то
време био језик образованих људи. Тако се као најважније године Вукове
реформе истичу 1818., 1836., 1839., 1847. и 1852.
\end{biografija}
% ------------------------------------------------------------------------------

\end{document} 